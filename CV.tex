\documentclass[10.5pt, oneside]{article}   	% use "amsart" instead of "article" for AMSLaTeX format
\usepackage[top=3cm, bottom=3cm, left=2cm, right=2cm]{geometry}                		% See geometry.pdf to learn the layout options. There are lots.
\geometry{a4paper}                   		% ... or a4paper or a5paper or ... 
\usepackage[parfill]{parskip}    		% Activate to begin paragraphs with an empty line rather than an indent
\usepackage{graphicx}				% Use pdf, png, jpg, or eps§ with pdflatex; use eps in DVI mode
								% TeX will automatically convert eps --> pdf in pdflatex		
\usepackage{amssymb}
\usepackage{titlesec}				% Can change section, subsection etc styles
\usepackage{enumitem}				% Can change list styles
\usepackage{multicol}				% Multicolumn
\usepackage{hyperref}
\usepackage{orcidlink}
\usepackage[UKenglish]{isodate}
\usepackage{fontawesome5}
\usepackage{fancyhdr}
\pagestyle{fancy}

\titleformat*{\section}{\large\bfseries}
\titlespacing*{\section}{0pt}{0.5\baselineskip}{0.1\baselineskip}
%\pagenumbering{gobble}
\setlist{itemsep=0pt}

\definecolor{Sectioncolour}{HTML}{4472C4}
\definecolor{RGcolour}{HTML}{5ECDB0}
\definecolor{LIcolour}{HTML}{2D64BC}
\definecolor{Orcidcolour}{HTML}{AECD54}

\let\oldheadrule\headrule% Copy \headrule into \oldheadrule
\renewcommand{\headrule}{\color{Sectioncolour}\oldheadrule}% Add colour to \headrule
\renewcommand{\headrulewidth}{0.8pt}

\usepackage[backend=bibtex,
style=ieee,
sorting=ydnt,
citestyle=numeric-comp
]{biblatex} 
\addbibresource{bib}

\chead{\color{Sectioncolour} \Large\bf Oliver Long}
%\lfoot{Updated \today.} 

\begin{document}
 
%----------------------------------------------------------------------------------------
%	ICONS
%----------------------------------------------------------------------------------------
 
\begin{center}
\huge
\href{mailto:oliver.long@aei.mpg.de}{\faIcon{envelope}} \:
\textcolor{Orcidcolour}{\href{https://orcid.org/0000-0002-3897-9272}{\faIcon{orcid}}} \:
\textcolor{RGcolour}{\href{https://www.researchgate.net/profile/Oliver-Long-3}{\faIcon{researchgate}}} \:
\textcolor{LIcolour}{\href{https://www.linkedin.com/in/oliverflong/}{\faIcon{linkedin}}} \:
\href{https://scholar.google.com/citations?user=92pSUO0AAAAJ&hl=en}{\faIcon{graduation-cap}} \:
\href{https://oliverlong.info}{\faIcon{info}}
\end{center}

 \vspace{1mm}

%----------------------------------------------------------------------------------------
%	RESEARCH SECTION
%----------------------------------------------------------------------------------------

{\color{Sectioncolour}
\section*{Research experience} 
\vspace{-3mm}
\noindent\rule{\linewidth}{0.6pt}}

\textbf{Postdoctoral Researcher}, Max Planck Institute for Gravitational Physics \hfill Nov 2022 --\\ \vspace{-5mm}
\begin{itemize}
\item Used numerical data to perform a resummation of weak-field expansions of the scattering angle due to a point-particle source to extend their validity into the strong-field regime \cite{Long:2024ltn}.
\item Extended the Spectral Einstein Code (\href{https://www.black-holes.org/for-researchers/spec}{\textsc{SpEC}}) to be capable of accurately and efficiently calculating hyperbolic binary black hole encounters \cite{Mendes:2025gov,Long:2025nmj,Scheel:2025jct}.
\end{itemize}

\textbf{Postdoctoral Research Fellow}, University of Southampton \hfill Jun 2022 -- Oct 2022\\ 
\vspace{-5mm}
\begin{itemize}
\item Used numerical data to extract high-order weak-field expansions of the scattering angle due to a point-particle source \cite{Barack:2023oqp}.
\end{itemize}

\textbf{Ph.D.\ Student}, University of Southampton \hfill Sep 2018 -- May 2022\\ 
\vspace{-5mm}
\begin{itemize}
\item Developed and implemented a numerical method for calculating perturbations due to a point-particle source in black hole perturbation theory \cite{Barack:2022pde,Long:2021ufh}.
\item Performed the first-ever calculation of the post-geodesic correction to the scattering angle of a point-particle in a black hole spacetime \cite{Barack:2022pde}.
\item Derived analytic formulae for scattering geodesics in a black hole spacetime \cite{Long:2021ufh}. Implemented the formulae in the KerrGeodesics package of the Black Hole Perturbation Toolkit \cite{BHPT:prep}.
\end{itemize}

%----------------------------------------------------------------------------------------
%	TEACHING SECTION
%----------------------------------------------------------------------------------------

{\color{Sectioncolour}
\section*{Teaching experience}
\vspace{-3mm}
\noindent\rule{\linewidth}{0.6pt}}

\textbf{Postgraduate Student Demonstrator}, University of Southampton \hfill Sep 2018 -- Jan 2022\\ \vspace{-5mm}
\begin{itemize}
\item Numerical Methods in Python: Run workshops based on problem sheets and coursework.
\item Maths for Physicists: Run classes demonstrating mathematical techniques and coursework marking.
\item Maths for Engineers: Run drop-in workshops, marking of exams, and exam writing.
\end{itemize}

%----------------------------------------------------------------------------------------
%	EDUCATION SECTION
%----------------------------------------------------------------------------------------

{\color{Sectioncolour}
\section*{Education}
\vspace{-3mm}
\noindent\rule{\linewidth}{0.6pt}}

\textbf{Ph.D.\ in Mathematical Sciences}, University of Southampton \hfill Sep 2018 -- Apr 2022 \\
\vspace{-5mm}
\begin{itemize}
\item Thesis title: Self-force in hyperbolic black hole encounters.
\item Advisor: Prof. Leor Barack
\end{itemize}
\textbf{MPhys in Physics}, The University of Manchester \hfill Sep 2014 -- Jun 2018 \\
\vspace{-5mm}
\begin{itemize}
\item Project title: Constraints on the neutrino sector using current and future cosmological data.
\item Advisor: Dr Eleonora Di Valentino
\end{itemize}

%----------------------------------------------------------------------------------------
%	CODES SECTION
%----------------------------------------------------------------------------------------

% {\color{Sectioncolour}
%\section*{Code contributions}
%\vspace{-3mm}
%\noindent\rule{\linewidth}{0.6pt}}
%
%\textbf{Time Domain Scalar Self-Force} \hfill \texttt{C++} and \texttt{Mathematica} \\
%\vspace{-5mm}
% \begin{itemize}
%\item Numerical solving of the Klein-Gordon equation for a point particle source on generic orbits.
%\item Designed and implemented by myself.
%\end{itemize}
%
%\textbf{Spectral Einstein Code} (SpEC) \hfill \texttt{C++}, \texttt{Perl}, and \texttt{Python} \\
%\vspace{-5mm}
% \begin{itemize}
%\item Numerical solving of the Einstein field equations for binary systems.
%\item Contributions include adaptive mesh refinement for hyperbolic orbits and diagnostic tools.
%\item Part of the Simulating Extreme Spacetimes (SXS) collection of codes.
%\end{itemize}
%
%\textbf{pySEOBNR} \hfill  \texttt{Python} and \texttt{Cython}\\
%\vspace{-5mm}
% \begin{itemize}
%\item Numerical solving of binary black hole dynamics using Effective One Body (EOB) methods.
%\item Contributions include extension of the code to be able to calculate hyperbolic encounters.
%\end{itemize}
%
%\textbf{Kerr Geodesics} \hfill \texttt{Mathematica} \\
%\vspace{-5mm}
% \begin{itemize}
%\item Package to calculate time-like geodesics around a Kerr black hole. 
%\item Contributions include extension of the package to be able to calculate hyperbolic orbits and unit tests.
%\item Part of The Black Hole Perturbation Toolkit.
%\end{itemize}

%----------------------------------------------------------------------------------------
%	PRIZES SECTION
%----------------------------------------------------------------------------------------

{\color{Sectioncolour}
\section*{Prizes and funding awards}
\vspace{-3mm}
\noindent\rule{\linewidth}{0.6pt}}

\textbf{Seal of Excellence}, European Commission Horizon Europe. \hfill 2025 \\
\vspace{-5mm}
\begin{itemize}
\item Project proposal: Gravitational Waves from Hyperbolic Encounters (GWHypE).
\item Recognised as a highly rated proposal for the Marie Skłodowska-Curie Actions Postdoctoral Fellowships.
\end{itemize}
\textbf{Doctoral Prize Fellowship}, Engineering and Physical Sciences Research Council. \hfill 2022 \\
\vspace{-5mm}
\begin{itemize}
\item Funding for the Postdoctoral Research Fellow position at the University of Southampton.
\item Grant number: EP/T517859/1
\end{itemize}
\textbf{Best Student Talk Runner Up}, 25th Capra Meeting on Radiation Reaction in GR. \hfill Jun 2022 \\
 
%----------------------------------------------------------------------------------------
%	PUBLICATIONS SECTION
%----------------------------------------------------------------------------------------

\newpage

{\color{Sectioncolour}
\section*{Publications {\rm (total: 12)}}
\vspace{-3mm}
\noindent\rule{\linewidth}{0.6pt}}

\nocite{*}

\vspace{-5mm}
\printbibliography[notkeyword={prep},title={~}]

% {\color{Sectioncolour} 
% \section*{Articles in preparation {\rm (total: 2)}}
% \vspace{-3mm}
% \noindent\rule{\linewidth}{0.6pt}}

% \vspace{-5mm}
% \printbibliography[keyword={prep},title={~}]


%----------------------------------------------------------------------------------------
%	PRESENTATIONS SECTION
%----------------------------------------------------------------------------------------
 
  {\color{Sectioncolour}
\section*{Invited seminars {\rm (total: 3)}}
\vspace{-3mm}
\noindent\rule{\linewidth}{0.6pt}}
\begin{itemize}
\item ``Black hole scattering in the strong-field regime: Merging post-Minkowskian theory with numerical methods", WQFT Seminar, Humboldt University, 25th November 2024.
\item ``Black hole scattering in the strong-field regime: Merging post-Minkowskian theory with numerical methods", Gravity Seminar, University of Southampton, 7th November 2024.
\item ``Black hole scattering in the strong-field regime: Merging post-Minkowskian theory with numerical methods", Séminaire Amplitudes et Gravitation sur l'Yvette, Institut des Hautes Études Scientifiques, 16th October 2024.
\end{itemize} 

  {\color{Sectioncolour}
\section*{Invited talks {\rm (total: 8)}}
\vspace{-3mm}
\noindent\rule{\linewidth}{0.6pt}}
\begin{itemize}
\item ``Applications of numerical self-force scattering simulations", 2nd Annual Workshop on Self-Force and Amplitudes, University of Southampton, 10th September 2025.
\item ``Highly accurate simulations of asymmetric black-hole scattering and cross validation of effective-one-body models", EOB@Work25, Istituto Nazionale di Fisica Nucleare, 4th September 2025.
\item ``Putting the hype in hyperbolic black hole scattering", Mathematical Methods for the General Relativistic Two-body Problem, National University of Singapore, 14th August 2025.
\item ``Modelling of unbound binary black hole encounters", Fundamental Physics Meets Waveforms With LISA, Max Planck Institute for Gravitational Physics, 6th September 2024.
\item ``Comparing numeric and analytic methods for black hole scattering in unequal mass systems", Gravitational Self-Force and Scattering Amplitudes Workshop, The Higgs Centre for Theoretical Physics, 20th March 2024.
\newpage
\item ``Self-force meets post-Minkowskian in the scattering regime" Gravitational Waves meet Amplitudes in the Southern Hemisphere, International Center for Theoretical Physics South American Institute for Fundamental Research, 24th August 2023.
\item ``Hyperbolic self-force calculations within a hyperboloidal framework" Infinity on a Gridshell, Niels Bohr Institute, 10th July 2023.
\item ``Extraction of high-order post-Minkowskian results from scattering self-force calculations" QCD meets Gravity 2022, Universität Zürich, 13th December 2022.
\end{itemize} 

 {\color{Sectioncolour}
\section*{Presentations {\rm (total: 14)}}
\vspace{-3mm}
\noindent\rule{\linewidth}{0.6pt}}
\begin{itemize}
\item ``Highly accurate simulations of asymmetric black-hole scattering and cross validation of effective-one-body models", New Frontiers of Numerical Relativity, Universitat de les Illes Balears, 22nd July 2025.
\item ``Highly accurate simulations of asymmetric black-hole scattering and cross validation of effective-one-body models", 24th International Conference on General Relativity and Gravitation and 16th Edoardo Amaldi Conference on Gravitational Waves, Glasgow, 15th July 2025.
\item ``Hyperbolic Binary Black Hole encounters with the Spectral Einstein Code", NR Community Call, 2nd December 2024.
\item ``Hyperbolic Binary Black Hole encounters with \texttt{SpEC}" Simulating Extreme Spacetimes with \texttt{SpEC} and \texttt{SpECTRE}, The Institute for Computational and Experimental Research in Mathematics, Brown University, 7th August 2024.
\item ``Double the hype: Hyperboloidal framework for self-force in hyperbolic black hole encounters" 27th Capra Meeting on Radiation Reaction in General Relativity, National University of Singapore, 17th June 2024.
\item ``Comparison of post-Minkowskian and self-force expansions: Scattering in a scalar charge toy model" 26th Capra Meeting on Radiation Reaction in General Relativity, Niels Bohr Institute, 4th July 2023.
\item ``Self-force in hyperbolic black hole encounters" LISA Symposium XIV, 25th -- 29th July 2022. \href{https://www.youtube.com/watch?v=p2-RgYB6Jhk}{\faIcon{link}}
\item ``Self-force in hyperbolic black hole encounters" 23rd International Conference on General Relativity and Gravitation, Chinese Academy of Science via Zoom, 5th July 2022. \href{https://www.koushare.com/video/videodetail/30159}{\faIcon{link}}
\item ``Self-force in hyperbolic black hole encounters" 25th Capra Meeting on Radiation Reaction in General Relativity, University College Dublin, 22nd June 2022. \href{https://oliverlong.info/talks/capra25}{\faIcon{link}}
\item ``Self-force in hyperbolic binary-black-hole encounters" BritGrav22, University of Glasgow via Zoom, \\ 4th April 2022. \href{https://www.youtube.com/watch?v=zZEblkjb5IM}{\faIcon{link}}
\item ``Time-domain metric reconstruction for hyperbolic scattering" 24th Capra Meeting on Radiation Reaction in General Relativity, Perimeter Institute via Zoom, 10th June 2021. \href{https://pirsa.org/21060058}{\faIcon{link}}
\item ``Towards a self-force calculation of the scatter angle in hyperbolic encounters" BritGrav21, University College Dublin via Zoom, 13th April 2021. \href{https://oliverlong.info/talks/britgrav21}{\faIcon{link}}
\item``Towards a self-force calculation of the scatter angle in hyperbolic encounters" LISA Symposium XIII, 1st -- 3rd October 2020. \href{https://lisasymposium13.lisamission.org/presentations/i0xMnRFWi7WKbO5f01caGDH0zPK2/7qz7uYC3qHuzj9AsC49h}{\faIcon{link}}
\item ``Towards a self-force calculation of the scatter angle in hyperbolic encounters" 23rd Capra Meeting on Radiation Reaction in General Relativity, University of Texas at Austin via Zoom, 24th June 2020. \href{https://www.youtube.com/watch?v=HB-Rw5kRUfg&t=11311s}{\faIcon{link}}
\end{itemize} 
 
{\color{Sectioncolour}
\section*{Posters {\rm (total: 1)}}
\vspace{-3mm}
\noindent\rule{\linewidth}{0.6pt}}
\begin{itemize}
\item ``Time-domain metric reconstruction using the Hertz potential" 3rd meeting of the GWVerse COST action, Institute for Fundamental Physics of the Universe, International School for Advanced Studies, 13th -- 16th January 2020.
\end{itemize} 

%----------------------------------------------------------------------------------------
%	EVENTS ATTENDED SECTION
%----------------------------------------------------------------------------------------
 
{\color{Sectioncolour}
\section*{Other events attended {\rm (total: 11)}}
\vspace{-3mm}
\noindent\rule{\linewidth}{0.6pt}}
\begin{itemize}
\item Black Hole Perturbation Toolkit Workshop (via Zoom), The Institute for Computational and Experimental Research in Mathematics, Brown University, 25th -- 27th July 2022.
\item From Scattering Amplitudes to Gravitational-Wave Predictions for Compact Binaries, Universität Zürich \& ETH Zürich, 4th -- 15th July 2022.
\item Advances and Challenges in Computational Relativity Workshop (Online), The Institute for Computational and Experimental Research in Mathematics, Brown University, 14th -- 18th September 2020.
\item Black Hole Perturbation Toolkit Workshop (Online), Astronomical Institute of the Academy of Sciences of the Czech Republic, 25th -- 27th May 2020.
\item Kavli RISE Summer School on Gravitational Waves, University of Cambridge, 23rd -- 27th September 2019.
\item 22nd International Conference on General Relativity and Gravitation and 13th Edoardo Amaldi Conference on Gravitational Waves, Palau de congressos de Valencia, 8th -- 12th July 2019.
\item 22nd Capra Meeting on Radiation Reaction in General Relativity, Centro Brasileiro de Pesquisas Físicas, 17th -- 21st June 2019.
\item LISA Waveform Working Group Meeting, Max Planck Institute for Gravitational Physics (Albert Einstein Institute), 13th -- 15th May 2019.
\item BritGrav19, Durham University, 15th -- 16th April 2019.
\item Black Hole Perturbation Toolkit Workshop, University College Dublin, 19th -- 21st March 2019.
\end{itemize} 


 {\color{Sectioncolour}
\section*{Other community service}
\vspace{-3mm}
\noindent\rule{\linewidth}{0.6pt}}
\begin{itemize}
\item Member of the Simulating eXtreme Spacetimes (SXS) Social Media Team.  \hfill Aug 2024 --\\ \vspace{-5mm}
\item Member of the Capra community's Equality, Diversity and Inclusion (EDI) Team.  \hfill Jun 2024 --\\ \vspace{-5mm}
\item Reviewer for {\em Physical Review Letters} and {\em Physical Review D}.  \hfill Mar 2023 --\\ \vspace{-5mm}
\item Discussion chair for ``Scattering -- What? Why? How?", Fundamental Physics Meets Waveforms With LISA, Max Planck Institute for Gravitational Physics, 6th September 2024.
\item Discussion chair for ``Self-force meets Amplitudes", 26th Capra Meeting on Radiation Reaction in General Relativity, Niels Bohr Institute, 4th July 2023.
\end{itemize} 

%----------------------------------------------------------------------------------------
%	COLLABORATIONS SECTION
%----------------------------------------------------------------------------------------
{\color{Sectioncolour}
\section*{Research collaborations}
\vspace{-3mm}
\noindent\rule{\linewidth}{0.6pt}}

{\bf Simulating eXtreme Spacetimes (SXS) collaboration}  \hfill Nov 2022 -- \\
\vspace{-5mm}
\begin{itemize}
\item Contributor to the Spectral Einstein Code (\texttt{SpEC}).
\item Member of the SXS Social Media Team.
\end{itemize} 
{\bf The Black Hole Perturbation Toolkit}  \hfill Mar 2019 -- \\
\vspace{-5mm}
\begin{itemize}
\item Contributor to the KerrGeodesics package.
\end{itemize}
{\bf Laser Interforemeter Space Antenna (LISA) Consortium}  \hfill Oct 2018 -- \\
\vspace{-5mm}
\begin{itemize}
\item Member of the Waveform Working Group and LISA Early Career Scientists (LECS).
\end{itemize} 
{\bf LIGO Scientific Collaboration (LSC)}  \hfill Apr 2023 -- Nov 2023 \\
\vspace{-5mm}
\begin{itemize}
\item Member of the Waveform Working Group.
\end{itemize} 

%----------------------------------------------------------------------------------------
%	COMPUTING SECTION
%----------------------------------------------------------------------------------------

 {\color{Sectioncolour}
\section*{Computing experience}
\vspace{-3mm}
\noindent\rule{\linewidth}{0.6pt}}

\textbf{Advanced}: \texttt{C}/\texttt{C++}, \texttt{Python}, \texttt{Mathematica}, Git, Linux, MacOS, \LaTeX. \vspace{1.5mm} \\ 
\textbf{Intermediate}: Bash, OpenMP, Slurm, Paraview, Windows. \vspace{1.5mm}\\
\textbf{Some experience}: \texttt{Perl}, \texttt{Cython}, OpenMPI.

%Name referees
%{\color{Sectioncolour}
%\section*{References}
%\vspace{-3mm}
%\noindent\rule{\linewidth}{0.6pt}}
%Available upon request.
%\begin{itemize}
%\item Prof.\ L.\ Barack; PhD supervisor; University of Southampton; \href{mailto:L.Barack@soton.ac.uk}{L.Barack@soton.ac.uk}.
%\item Dr.\ A.\ Pound; Secondary PhD supervisor; University of Southampton; \href{mailto:A.Pound@soton.ac.uk}{A.Pound@soton.ac.uk}.
%\item Prof.\ B.\ Wardell; Research colleague; University College Dublin; \href{mailto:barry.wardell@ucd.ie}{barry.wardell@ucd.ie}.
%\item Dr.\ E.\ Di Valentino; MPhys project supervisor; University of Sheffield; e.divalentino@sheffield.ac.uk.
%\end{itemize}

\end{document}  