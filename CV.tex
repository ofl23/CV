\documentclass[10.5pt, oneside]{article}   	% use "amsart" instead of "article" for AMSLaTeX format
\usepackage[top=3cm, bottom=3cm, left=2cm, right=2cm]{geometry}                		% See geometry.pdf to learn the layout options. There are lots.
\geometry{a4paper}                   		% ... or a4paper or a5paper or ... 
\usepackage[parfill]{parskip}    		% Activate to begin paragraphs with an empty line rather than an indent
\usepackage{graphicx}				% Use pdf, png, jpg, or eps§ with pdflatex; use eps in DVI mode
								% TeX will automatically convert eps --> pdf in pdflatex		
\usepackage{amssymb}
\usepackage{titlesec}				% Can change section, subsection etc styles
\usepackage{enumitem}				% Can change list styles
\usepackage{multicol}				% Multicolumn
\usepackage{hyperref}
\usepackage{orcidlink}
\usepackage[UKenglish]{isodate}
\usepackage{fontawesome5}
\usepackage{fancyhdr}
\pagestyle{fancy}

\titleformat*{\section}{\large\bfseries}
\titlespacing*{\section}{0pt}{0.5\baselineskip}{0.1\baselineskip}
\pagenumbering{gobble}
\setlist{itemsep=0pt}

\definecolor{Sectioncolour}{HTML}{4472C4}
\definecolor{RGcolour}{HTML}{5ECDB0}
\definecolor{LIcolour}{HTML}{2D64BC}
\definecolor{Orcidcolour}{HTML}{AECD54}

\let\oldheadrule\headrule% Copy \headrule into \oldheadrule
\renewcommand{\headrule}{\color{Sectioncolour}\oldheadrule}% Add colour to \headrule
\renewcommand{\headrulewidth}{0.8pt}

\chead{\color{Sectioncolour} \Large\bf Oliver Long}
\lfoot{Updated \today.}

\begin{document}
 
%----------------------------------------------------------------------------------------
%	ICONS
%----------------------------------------------------------------------------------------
 
\begin{center}
\huge
\href{mailto:oliver.long@aei.mpg.de}{\faIcon{envelope}} \:
\textcolor{Orcidcolour}{\href{https://orcid.org/0000-0002-3897-9272}{\faIcon{orcid}}} \:
\textcolor{RGcolour}{\href{https://www.researchgate.net/profile/Oliver-Long-3}{\faIcon{researchgate}}} \:
\textcolor{LIcolour}{\href{https://www.linkedin.com/in/oliverflong/}{\faIcon{linkedin}}} \:
\href{https://scholar.google.com/citations?user=92pSUO0AAAAJ&hl=en}{\faIcon{graduation-cap}} \:
\href{https://oliverlong.info}{\faIcon{info}}
\end{center}

 \vspace{1mm}

%----------------------------------------------------------------------------------------
%	EDUCATION SECTION
%----------------------------------------------------------------------------------------

{\color{Sectioncolour}
\section*{Academic positions}
\vspace{-3mm}
\noindent\rule{\linewidth}{0.6pt}}

\textbf{Max Planck Institute for Gravitational Physics (Albert Einstein Institute) Potsdam} \\
\vspace{-5mm}
\begin{itemize}
\item Postdoctoral Researcher \hfill Nov 2022 -- Present
\end{itemize}
Modelling of hyperbolic orbits of binary black holes using Effective One-Body, Numerical Relativity and self-force methods.

\textbf{University of Southampton} \\
\vspace{-5mm}
\begin{itemize}
\item Doctoral Prize Research Fellow \hfill Aug 2022 -- Oct 2022
\item Doctoral Prize Senior Research Assistant \hfill Jun 2022 -- Jul 2022
\item Research Assistant \hfill Jan 2022 -- Mar 2022
\end{itemize}
Modelling of hyperbolic orbits of binary black holes with back-reaction in the extreme-mass-ratio limit using perturbation theory focusing on calculating the scalar self-force correction to the scatter angle as well as calculating the scatter angle to first-order in the mass ratio.

{\color{Sectioncolour}
\section*{Education}
\vspace{-3mm}
\noindent\rule{\linewidth}{0.6pt}}

\textbf{Ph.D. in Mathematical Sciences}, University of Southampton \hfill Sep 2018 -- Jun 2022 \\
\vspace{-5mm}
\begin{itemize}
\item Project title: Self-force in hyperbolic black hole encounters.
\item Supervisor: Prof.\ Leor Barack.
\item Project: Modelling of hyperbolic orbits of binary black holes with back-reaction in the extreme-mass-ratio limit using perturbation theory focusing on the derivation and development of a model to calculate the self-force correction to the scatter angle.
\item Teaching: Running of undergraduate problem classes and helping with assessment.
\end{itemize}

\textbf{MPhys in Physics}, The University of Manchester \hfill Sep 2014 -- Jun 2018 \\
\vspace{-5mm}
\begin{itemize}
\item Final degree grade: First-class honours with an average of $80.4\%$.
\item Modules: Gravitation, The Early Universe, Quantum Field Theory,  Electrodynamics, etc.
\item Master's project: Using Markov chain Monte Carlo methods on power spectra of the cosmic microwave background to resolve various tensions in the data through the use of different cosmological models.
\end{itemize} 

\textbf{A levels}, Hereford Sixth Form College \hfill Sep 2012 -- Jun 2014 \\
\vspace{-5mm}
\begin{itemize}
\item A2 levels: Physics (A*), Mathematics (A), Chemistry (A) and Biology (A).
\item AS levels: Further Mathematics (A).
\end{itemize} 

\textbf{GCSEs}, Lacon Childe School \hfill Sep 2007 -- Jun 2012\\
\vspace{-5mm}
\begin{itemize}
\item Eleven level 2 awards including GCSEs in English language and German.
\end{itemize} 

 {\color{Sectioncolour}
\section*{Computing}
\vspace{-3mm}
\noindent\rule{\linewidth}{0.6pt}}
 \begin{itemize}
\item Extensive experience with \texttt{Mathematica} including tensor algebra, data analysis and graphics. 
\item Extensive experience with \texttt{C++} including numerical calculations and data analysis.
\item Extensive experience with \texttt{Python} including data analysis and graphics. 
\item Extensive experience with Git.
\item Extensive experience with \LaTeX.
\end{itemize}

%----------------------------------------------------------------------------------------
%	PRIZES SECTION
%----------------------------------------------------------------------------------------

{\color{Sectioncolour}
\section*{Prizes}
\vspace{-3mm}
\noindent\rule{\linewidth}{0.6pt}}
\begin{itemize}
\item \textbf{Doctoral Prize}, Engineering and Physical Sciences Research Council. \hfill 2022
\item \textbf{Best Student Talk Runner Up}, 25th Capra Meeting on Radiation Reaction in GR. \hfill Jun 2022 
\end{itemize}
 
%----------------------------------------------------------------------------------------
%	PUBLICATIONS/PRESENTATIONS SECTION
%----------------------------------------------------------------------------------------

{\color{Sectioncolour}
\section*{Featured Publications}
\vspace{-3mm}
\noindent\rule{\linewidth}{0.6pt}}
\begin{itemize}
\item L. Barack, Z. Bern, E. Hermann, {\bf O. Long}, {\it et al}. Comparison of post-Minkowskian and self-force expansions: Scattering in a scalar charge toy model. {\it Phys.\ Rev.\ D}, {\bf 108}(024025), Jul 2023. \href{https://journals.aps.org/prd/abstract/10.1103/PhysRevD.108.024025}{\faIcon{link}}
\item L. Barack \& {\bf O. Long}. Self-force correction to the deflection angle in black-hole scattering: A scalar charge toy model. {\it Phys.\ Rev.\ D}, {\bf 106}(104031), Nov 2022. \href{https://journals.aps.org/prd/abstract/10.1103/PhysRevD.106.104031}{\faIcon{link}}
\item {\bf O. Long} \& L. Barack. Time-domain metric reconstruction for hyperbolic scattering. {\it Phys.\ Rev.\ D}, {\bf 104}(024014), Jul 2021. \href{https://journals.aps.org/prd/abstract/10.1103/PhysRevD.104.024014}
{\faIcon{link}}
\end{itemize} 

{\color{Sectioncolour} 
\section*{Other publications}
\vspace{-3mm}
\noindent\rule{\linewidth}{0.6pt}}
\begin{itemize}
\item M. Boschini, D. Gerosa, V. Varma, {\it et al}. Extending black-hole remnant surrogate models to extreme mass ratios. {\it arXiv:2307.03435}. \href{https://arxiv.org/abs/2307.03435}{\faIcon{link}}
\item L. J. Gomes Da Silva, R. Panosso Macedo, J. E. Thompson,  J. A. Valiente Kroon, {\it et al}. Hyperboloidal discontinuous time-symmetric numerical algorithm with higher order jumps for gravitational self-force computations in the time domain. {\it arXiv:2306.13153}. \href{https://arxiv.org/abs/2306.13153}{\faIcon{link}}
\item L. J. Gomes Da Silva, R. Panosso Macedo, {\bf O. Long} \& J. A. Valiente Kroon. Numerical Algorithm for the Computation of the Scalar Self-Force on a Charged Particle on a Schwarzschild background in the Time Domain. ({\it in preparation}).
\item M. Boyle, K. Chatziioannou, {\it et al}. The SXS Collaboration catalog of binary black hole simulations II. ({\it in preparation}).
\item N. Warburton, B. Wardell, {\it et al}. The Black Hole Perturbation Toolkit. ({\it in preparation}).
\item S. Akçay, E. Barausse, {\it et al}. LISA Waveform Modelling Whitepaper. ({\it in preparation}).
\end{itemize} 

%----------------------------------------------------------------------------------------
%	PRESENTATIONS SECTION
%----------------------------------------------------------------------------------------
 
  {\color{Sectioncolour}
\section*{Invited talks}
\vspace{-3mm}
\noindent\rule{\linewidth}{0.6pt}}
\begin{itemize}
\item``Self-force meets post-Minkowskian in the scattering regime" Gravitational Waves meet Amplitudes in the Southern Hemisphere, International Center for Theoretical Physics South American Institute for Fundamental Research, 24th August 2023.
\item``Hyperbolic self-force calculations within a hyperboloidal framework" Infinity on a Gridshell, Niels Bohr Institute, 10th July 2023.
\item``Extraction of high-order post-Minkowskian results from scattering self-force calculations" QCD meets Gravity 2022, Universität Zürich, 13th December 2022.
\end{itemize} 
 
 {\color{Sectioncolour}
\section*{Conference presentations}
\vspace{-3mm}
\noindent\rule{\linewidth}{0.6pt}}
\begin{itemize}
\item ``Comparison of post-Minkowskian and self-force expansions: Scattering in a scalar charge toy model" 26th Capra Meeting on Radiation Reaction in General Relativity, Niels Bohr Institute, 4th July 2023.
\item``Self-force in hyperbolic black hole encounters" LISA Symposium XIV, 25th -- 29th July 2022. \href{https://www.youtube.com/watch?v=p2-RgYB6Jhk}{\faIcon{link}}
\item ``Self-force in hyperbolic black hole encounters" 23rd International Conference on General Relativity and Gravitation, Chinese Academy of Science via Zoom, 5th July 2022. \href{https://www.koushare.com/video/videodetail/30159}{\faIcon{link}}
\item ``Self-force in hyperbolic black hole encounters" 25th Capra Meeting on Radiation Reaction in General Relativity, University College Dublin, 22nd June 2022. \href{https://oliverlong.info/talks/capra25}{\faIcon{link}}
\item ``Self-force in hyperbolic binary-black-hole encounters" BritGrav22, University of Glasgow via Zoom, \\ 4th April 2022. \href{https://www.youtube.com/watch?v=zZEblkjb5IM}{\faIcon{link}}
\item ``Time-domain metric reconstruction for hyperbolic scattering" 24th Capra Meeting on Radiation Reaction in General Relativity, Perimeter Institute via Zoom, 10th June 2021. \href{https://pirsa.org/21060058}{\faIcon{link}}
\item ``Towards a self-force calculation of the scatter angle in hyperbolic encounters" BritGrav21, University College Dublin via Zoom, 13th April 2021. \href{https://oliverlong.info/talks/britgrav21}{\faIcon{link}}
\item``Towards a self-force calculation of the scatter angle in hyperbolic encounters" LISA Symposium XIII, 1st -- 3rd October 2020. \href{https://lisasymposium13.lisamission.org/presentations/i0xMnRFWi7WKbO5f01caGDH0zPK2/7qz7uYC3qHuzj9AsC49h}{\faIcon{link}}
\item ``Towards a self-force calculation of the scatter angle in hyperbolic encounters" 23rd Capra Meeting on Radiation Reaction in General Relativity, University of Texas at Austin via Zoom, 24th June 2020. \href{https://www.youtube.com/watch?v=HB-Rw5kRUfg&t=11311s}{\faIcon{link}}
\end{itemize} 
 
{\color{Sectioncolour}
\section*{Conference posters}
\vspace{-3mm}
\noindent\rule{\linewidth}{0.6pt}}
\begin{itemize}
\item ``Time-domain metric reconstruction using the Hertz potential" 3rd meeting of the GWVerse COST action, Institute for Fundamental Physics of the Universe, International School for Advanced Studies, 13th -- 16th January 2020.
\end{itemize} 

%----------------------------------------------------------------------------------------
%	EVENTS ATTENDED SECTION
%----------------------------------------------------------------------------------------
 
{\color{Sectioncolour}
\section*{Other events attended}
\vspace{-3mm}
\noindent\rule{\linewidth}{0.6pt}}
\begin{itemize}
\item Black Hole Perturbation Toolkit Workshop (via Zoom), The Institute for Computational and Experimental Research in Mathematics, Brown University, 25th -- 27th July 2022.
\item From Scattering Amplitudes to Gravitational-Wave Predictions for Compact Binaries, Universität Zürich \& ETH Zürich, 4th -- 15th July 2022.
\item Advances and Challenges in Computational Relativity Workshop (Online), The Institute for Computational and Experimental Research in Mathematics, Brown University, 14th -- 18th September 2020.
\item Black Hole Perturbation Toolkit Workshop (Online), Astronomical Institute of the Academy of Sciences of the Czech Republic, 25th -- 27th May 2020.
\item Kavli RISE Summer School on Gravitational Waves, University of Cambridge, 23rd -- 27th September 2019.
\item 22nd International Conference on General Relativity and Gravitation and 13th Edoardo Amaldi Conference on Gravitational Waves, Palau de congressos de Valencia, 8th -- 12th July 2019.
\item 22nd Capra Meeting on Radiation Reaction in General Relativity, Centro Brasileiro de Pesquisas Físicas, 17th -- 21st June 2019.
\item LISA Waveform Working Group Meeting, Max Planck Institute for Gravitational Physics (Albert Einstein Institute), 13th -- 15th May 2019.
\item BritGrav19, Durham University, 15th -- 16th April 2019.
\item Black Hole Perturbation Toolkit Workshop, University College Dublin, 19th -- 21st March 2019.
\end{itemize} 

%----------------------------------------------------------------------------------------
%	COLLABORATIONS SECTION
%----------------------------------------------------------------------------------------

{\color{Sectioncolour}
\section*{Research collaborations}
\vspace{-3mm}
\noindent\rule{\linewidth}{0.6pt}}
{\bf Laser Interforemeter Space Antenna (LISA) Consortium}  \hfill Oct 2018 -- Present \\
\vspace{-5mm}
\begin{itemize}
\item Member of the Waveform Working Group and LISA Early Career Scientists (LECS).
\end{itemize} 
{\bf The Black Hole Perturbation Toolkit}  \hfill Mar 2019 -- Present \\
\vspace{-5mm}
\begin{itemize}
\item Contributor to the KerrGeodesics package.
\end{itemize}
{\bf Simulating eXtreme Spacetimes (SXS) collaboration}  \hfill Nov 2022 -- Present \\
\vspace{-5mm}
\begin{itemize}
\item Contributor to the Spectral Einstein Code (SpEC).
\end{itemize} 
{\bf LIGO Scientific Collaboration (LSC)}  \hfill Apr 2023 -- Present \\
\vspace{-5mm}
\begin{itemize}
\item Contributor to the pySEOBNR code.
\end{itemize} 

%Name referees
{\color{Sectioncolour}
\section*{References}
\vspace{-3mm}
\noindent\rule{\linewidth}{0.6pt}}
Available upon request.
%\begin{itemize}
%\item Prof.\ L.\ Barack; PhD supervisor; University of Southampton; \href{mailto:L.Barack@soton.ac.uk}{L.Barack@soton.ac.uk}.
%\item Dr.\ A.\ Pound; Secondary PhD supervisor; University of Southampton; \href{mailto:A.Pound@soton.ac.uk}{A.Pound@soton.ac.uk}.
%\item Prof.\ B.\ Wardell; Research colleague; University College Dublin; \href{mailto:barry.wardell@ucd.ie}{barry.wardell@ucd.ie}.
%\item Dr.\ E.\ Di Valentino; MPhys project supervisor; University of Sheffield; e.divalentino@sheffield.ac.uk.
%\end{itemize}

\end{document}  